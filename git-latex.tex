\documentclass{ltxdoc}
\usepackage[english]{babel}
\usepackage[nonewpage]{imakeidx}
\usepackage[authors,titlepage,email]{git}

\usepackage{listings}

\usepackage{calc}
\usepackage{multicol}
\usepackage{tabularx}
\usepackage{xcolor}
\usepackage{textcomp}

\usepackage[orig,english]{isodate}

\title{\texttt{Git} \LaTeX{} package}

\def\cmd{\lstinline[style=TeX]}
\def\ifempty#1{\def\temp{#1}\ifx\temp\empty}
\newcommand\largs[1]{{\ttfamily (}\textlangle\textit{#1}\textrangle{\ttfamily{)}}}
%\def\cmd{\lstinline[style=LaTeX]}
\newcommand\MentionLua[1]{\lstinline[language={[5.3]Lua}]{git.#1}\index[lua]{#1}}
\newcommand\DescribeLua[2][]{\ifempty{#1}\quad\else\largs{#1}\fi\marginpar{\hfill\MentionLua{#2}}}
\newcommand\DescribeLuaFunc[2][]{\ifempty{#1}{\ttfamily()}\else\largs{#1}\fi\marginpar{\hfill\MentionLua{#2}}}

%\CodelineIndex
%\EnableCrossrefs
%\makeindex[name=lua,title={Lua {\ttfamily git.}\textlangle {\itshape functions}\textrangle},columns=1]
%\makeindex[name=pkgopts,title={Package Options},columns=1]
%\makeindex[options=-s gind.ist,title={Macro's},columns=1]
%\indexsetup{level=\subsection*}

\usepackage{hyperref}

\begin{document}
    \RecordChanges

    \maketitle

    \begin{abstract}
        This project aims to display git project information in PDF documents.
        It's mostly written in Lua for executing the \texttt{git} commands.
        For \LaTeX{} it provides a set of standard macro's for displaying basic information or setting the project directory, and a set of advanced macro's for commits and tags.
        The advanced macro's also have iterating variants in order to generate changelogs or display multiple authors.
    \end{abstract}

    \begin{multicols}{2}
    \tableofcontents
    \lstlistoflistings
    \end{multicols}
    \clearpage

    \section{Usage}
    For the package to work one should work, and only work, with Lua\TeX{}.
    Another prerequisite is that there is an available git repository either in the working directory, or somewhere else on your machine (see section~\ref{sub:tex-basic} for \LaTeX{}, or~\ref{sub:lua-basic} for \texttt{Lua}).

    \subsection{Git}
    For this package to work at a minimum, there has to be an initialized Git repository, and preferably, at least with one commit.
    For example, the following minimal example should do the trick already:
    \begin{lstlisting}[style=bash,frame=single,caption={Minimal Git setup}]
mkdir my_project
cd my_project
echo "# My Project" > README.md
git init && git commit -am "Init"
    \end{lstlisting}

    Then in order for the changelog to work, the project needs to contain either `lightweight-' or `annotated' tags.
    The main difference is that a lightweight tag takes no extra options, for example: \lstinline[style=Bash]{git tag 0.1}.
    See listing~\ref{lst:scenario} for more examples on authoring and versioning with \texttt{git}.

    \subsection{Lua\LaTeX{}}

    For generating the document with \LaTeX{} one should make use of \lstinline|lualatex|.
    For example, when having the main file `\texttt{main.tex}':
    \begin{lstlisting}[style=bash,frame=single,caption={Generating the document with \LaTeX{}}]
# Generate once
lualatex -shell-escape main
# Generate and keep watching with LaTeXMK
latexmk -pvc -lualatex -shell-escape main
    \end{lstlisting}
    Note that in both cases option `-shell-escape' is required.
    This is required for issuing \texttt{git} via the commandline.

    \subsection{Lua\TeX{}}
    If you'd like to do it from Lua directly, one can require it as follows:
    \begin{lstlisting}[language={[5.3]Lua},frame=single,caption={Importing git-latex from Lua}]
local git = require('git-latex')
-- Print project version to the log
texio.write_nl('My Project v' .. git.version())
    \end{lstlisting}


    \section{LaTeX Interface}

    \subsection{Package Options}

    \begin{center}
        \cmd{\usepackage}%
        \oarg{lastcommit\textbar tags, author\textbar authors, alpha\textbar contrib, titlepage}%
        \texttt{\textbraceleft git\textbraceright}\index[pkgopts]{git(.sty)}
    \end{center}

    \subsection{Basic Macro's}\label{sub:tex-basic}

    \noindent
    \DescribeMacro{\gitdirectory}
    \DescribeMacro{\gitunsetdirectory}

    \noindent
    \DescribeMacro{\gitversion}

    \noindent
    \DescribeMacro{\gitauthor}

    \noindent
    \DescribeMacro{\gitemail}

    \noindent
    \DescribeMacro{\gitdate}

    \subsection{Multiple Authors}

    \noindent
    \DescribeMacro{\forgitauthor}
    \oarg{conjunction} maybe always accept 2 args for doAuthor

    \subsection{Commits}

    \noindent
    \DescribeMacro{\gitcommit}\oarg{format}\marg{csname}\marg{revision}\\
    For displaying commit data \cmd{\gitcommit} can be used.
    The optional \texttt{format} takes variables separated by a comma.
    The default \texttt{format} is \texttt{h,an,ae,as,s,b}.
    The \texttt{csname} is a user defined command accepting every variable as argument.\\

    \newlength\xample
    \setlength\xample{6cm-5pt}
    \newlength\xamplesep
    \setlength\xamplesep{5pt}
    \noindent
    \begin{minipage}[t]{\linewidth-\xample-\xamplesep}
        \noindent
        For example:
        \begin{lstlisting}[style=tex,numbers=left]
\newcommand{\formatcommit}[3]%
{#1, by #2 on \printdate{#3}}

\gitcommit[s,an,as]%
{formatcommit}{75dc036}
        \end{lstlisting}
    \end{minipage}\hfill%
    \begin{minipage}[t]{\xample}
        \noindent
        Results in:\\

        \noindent
        \setlength{\fboxsep}{5pt}%
        \newcommand\formatcommit[3]{#1, by #2 on \printdate{#3}}%
        \fbox{\parbox{\linewidth-2\fboxsep}{\gitcommit[s,an,as]{formatcommit}{75dc036}\footnotemark}}
    \end{minipage}\\
    \footnotetext{\texttt{\textbackslash printdate} from \texttt{isodate}: \url{https://www.ctan.org/pkg/isodate}}

    Consult \texttt{man git-log} for possible format variables and omit the \% for every variable.\\

    \noindent
    \DescribeMacro{\forgitcommit}\oarg{format}\marg{csname}\marg{rev\_spec}\\
    For displaying multiple commits the~\cmd{\forgitcommit} is used, which has the same arguments as \cmd{\gitcommit}, but only this time the \texttt{csname} is executed for every commit.
    The last argument \texttt{rev\_spec} this time, however, can have no argument or a sequence.\\
    \noindent\setlength\xample{5.7cm}
    \begin{minipage}[t]{\linewidth-\xample-\xamplesep}
        \noindent
        For example:
        \noindent
        \begin{lstlisting}[style=TeX,numbers=left]
\newcommand{\formatcommits}[2]%
{\item #1\\\quad —#2}

\begin{itemize}
    \forgitcommit[s,an]%
    {formatcommits}%
    {75dc036...e51c481}
\end{itemize}
        \end{lstlisting}
    \end{minipage}\hfill%
    \begin{minipage}[t]{\xample}
        \noindent
        Results in:\\

        \noindent
        \newcommand\formatcommits[2]{\item #1\\\quad —#2}%
        \setlength{\fboxsep}{0pt}%
        \fbox{\parbox{\linewidth}{%
            \begin{itemize}
                \forgitcommit[s,an]
                {formatcommits}
                {75dc036...e51c481}
            \end{itemize}
        }}
    \end{minipage}\\

    \subsection{Tags}

    \noindent
    \DescribeMacro{\forgittag}

    \noindent
    \DescribeMacro{\forgittagcommit}


    \section{Lua Interface}

    \subsection{Basic Functions}\label{sub:lua-basic}
    \DescribeLuaFunc[new\_directory]{directory}\\
    \DescribeLuaFunc{set\_date}\\
    \DescribeLuaFunc{version}\\
    \DescribeLuaFunc{author}\\
    \DescribeLuaFunc{email}\\
    \DescribeLuaFunc{commit\_date}\\

    \subsection{Multiple Authors}
    \DescribeLuaFunc[csname, conj, append\_email, sort]{for\_author}\\

    \subsection{Commits and Tags}
    \DescribeLuaFunc{tag\_list}\\
    \DescribeLuaFunc[csname, rev]{commit}\\
    \DescribeLuaFunc[csname, revspec]{for\_commit}\\
    \DescribeLuaFunc[csname]{for\_tag}\\
    \DescribeLuaFunc[csname\_tag, csname\_commit, after\_commits]{for\_tag\_and\_commit}\\
%
    % todo .cfg layered pages & code over ltxdocs
%    % Remove caching
%    \gitdirectory{../../tex-workspace}
%    \forgitauthor[, ]\\[2em]
%    \gitunsetdirectory
%    \gitdirectory{../../ginvoice}
%    Ginvoice
%    \gitversion\\
%    \directlua{git.tag_list()}
%
%    \newcommand\formattag[6]{\item #1, #4\\—#2 \textlangle#3\textrangle\begin{itemize}}
%    \newcommand\formatcommit[5]{\item \textbf{#1}\quad #2\\#3, #4\\[2em]#5}
%%    \newcommand\printchangelog[1]{\directlua{git.for_tag_and_commit('formattag', 'formatcommit', '#1')}}
%    \begin{itemize}
%        \gitcommit[h,s,as,an,b]{formatcommit}{75dc036}
%        \forgitcommit[h,s,as,an,b]{formatcommit}{}
%%        \forgitcommit[h,s,as,an]{nonexisting}{}
%%        \forgitcommit[h,s,as,an]{formatcommit}{e51c481...df76269}
%    \end{itemize}
%    \begin{itemize}
%        \printchangelog{\\end{itemize}}
%    \end{itemize}

    \section{Installation}

    \subsection{Building the Documentation}
    This documentation uses an example \texttt{project} which gets created by the \texttt{git-scenario.sh} script (see listing~\ref{lst:scenario}).
    It creates some commits having dates in the past and different authors set.
    Lastly it creates a `lightweight-' and `annotated' tag.

    To set up this scenario either use the \texttt{Makefile} or a bash shell directly.
    \begin{lstlisting}[style=bash,morekeywords={make,bash}]
# Using Bash
bash git-scenario.sh
# Using Make
make scenario
    \end{lstlisting}

    \lstinputlisting[style=bash,numbers=left,frame=single,label={lst:scenario},caption={git-scenario.sh},captionpos=t,morekeywords={git,alice,bob,charlie,mkdir,rm,curl,set\_author}]{git-scenario.sh}


    \gitunsetdirectory

%    \PrintIndex
%    \section*{Index}
%    \addcontentsline{toc}{section}{Index}
%    \begin{multicols}{2}
%        \printindex[pkgopts]
%        \printindex
%        \printindex[lua]
%    \end{multicols}
\end{document}
